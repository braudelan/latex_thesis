Organic matter is a key feature for soil health and fertility in both natural and managed ecosystems. The use of organic amendments (OAs) such as compost or animal manure is a common practice in agri-ecosystems that can increase soil organic matter stocks and enhance related soil properties such as  water and aeration management, and nutrient availability. OAs can be especially effective in arid and semi-arid climates, wher⎄e soils are often poor in SOM and soil fertility and where water resources are scarce. However, many other practices associated with conventional agricultural management can lead to extensive soil degradation, with serious consequences for farming enterprises and the entire ecosystem.        
        