% !TeX root = ../main.tex

\section{Benefits of soil organic matter}


	\gls{som} is a key factor in maintaining ecological services provided by soil ecosystems.
	\gls{som} contributes to soil fertility and health by retaining plant-available water and nutrients and promoting the formation of soil structure. \gls{som} is also consumed in the process of arable soil management as it releases needed nutrients and energy when it decomposes. Indeed soil organic matter is often highly beneficial when it decays and releases energy and nutrients \citep{lehmann2015, janzen2006}. Therefore it is important to explore  management practices that will allow both the accrual and simultaneous  decomposition of organic matter in a balanced ecosystem.

\section{Shift in paradigm regarding \gls{som} formation}

	 In the last  decades, there is an ongoing shift in  paradigm, concerning the nature of \gls{som} formation and retention in the soil. This shift is from what is sometimes referred to as the ‘humification’ model, in which the persistence of \gls{som} is mostly associated with the molecular structure of organic matter, to a more ecological framework (soil continuum model) that emphasizes the importance of environmental and biological factors in controlling the turnover of organic matter \citep{lehmann2015}. In the historical, \textit{humification} model, above-ground plant carbon inputs were regarded as the main source for \gls{som} formation. Stable organic matter was seen to comprise mainly selectively preserved plant inputs and de novo synthesis products like humic substances, whose chemical complexity and composition render them nearly inert relative to microbial degradation.\\
	 The emerging understanding is that the molecular structure of organic material does not necessarily determine its stability in soil. Rather, \gls{som} cycling is governed by multiple processes shaped by environmental conditions (such as physical heterogeneity)\citep{schmidt2011c}. This shift in paradigm was fueled by a number of novel discoveries regarding the nature of \gls{om} decomposition and stabilization in the soil. Firstly, the preservation of recalcitrant plant material as a basis for \gls{som} accumulation was repeatedly rebuked by evidence demonstrating the ability of soil microbes to rapidly decompose almost any biologically derived substance \citep{dungait2012, marschner2008} and the preferential stabilization (i.e longer turnover time) of labile plant products over recalcitrant materials has been demonstrated in a number of experiments \citep{cotrufo2013, kleber2011}. Concurrently, the existence of stable, large and complex molecular structures in the soil(i.e humification products) was substantially called into question\citep{kleber2010}. Finally the outlining of a number of previously under-recognized mechanisms for stabilization and preservation of \gls{om} in the soil was detailed in several papers, emphasizing the role of reduced accessibility for microbial decomposition (i.e. Physical protection inside aggregates and chemical bonding to mineral surfaces), as a primary determinant of \gls{om} stabilization and persistence\citep{lutzow2006, lutzow2008, ekschmitt2008}\.\\
	 This new framework for \gls{som} formation stresses the dynamic nature of \gls{som} management, wherein the preservation and enhancement of \gls{om} in a living soil (such as topsoil) depends not on the minimization of decomposition activity, for example by providing recalcitrant \gls{om} amendments or by targeting \gls{om} supply into areas of lower microbial activity (such as subsoil), but rather on the continuous supply of labile organic inputs, inducing high levels of microbial activity, which is able to facilitate the efficient formation of organo-mineral bonds and the protection of \gls{om} within aggregates\citep{dungait2012,barre2016,basler2015}.


\section{Microbial population as an essential component in \gls{om} cycling}

  	The soil microbial population is the major biological compartment responsible for decomposition and transformation of organic inputs \citep{thiet2006}. A growing body of evidence demonstrates that microbial products are the largest contributor to stable \gls{som} and that \gls{som} formation rates are positively and closely connected with the growth rates of microbial biomass \citep{kallenbach2016, kallenbach2015, ludwig2015, schurig2013, bradford2013}. Microbes drive \gls{som} formation through several mechanisms, both direct and indirect. Cell components of dead microbes are now considered a major source for stabilized \gls{om} \citep{kallenbach2015, liang2011, miltner2009} and it has been shown that these microbial remains are often preferentially concentrated in the more stable fractions of the soil \citep{ludwig2015}. \citet{liang2011} estimated that microbialy derived substances could make up more than 80\% of \gls{som}⁠.
%  	the soil amino acid signature has been found to correlate  with the  bacterial community structure
%  	\citep{moon2016}.
  	\citet{kiem2003} found polysaccharides, mainly those of microbial origin, are stabilized over the long-term within fine separates of arable soils.\\
  	Apart from being a major source of readily stabilized \gls{om}, micro-organisms are highly involved in the  modification of the soil environment, providing conditions that enhance the further decomposition and stabilization of additional \gls{om}.  For example, the production of extracellular substances by some micro-organisms is strongly related to the formation of aggregate structure and the mediation of water and oxygen concentrations, thus creating favorable growth environment and allowing for increased capacity for organic matter stabilization \citep{schimel2012} ⁠and as already mentioned, microbial remains form a large fraction of stabilized \gls{som} and have an important role in the formation of stable aggregates. Additionally, the release of organic acids by certain microbes has been shown to promote \gls{som} formation through increased mineral dissolution and mineral availability, thus increasing the potential for organo-mineral bonding \citep{yu2018}.



\section{Substrate Use Efficiency }
%todo force \gls{cue} to present long form 
%todo find a better term for 'neasure' in 'This \myRed{measure} a common indicator for overall substrate...'
	Microbes metabolize a wide variety of compounds to satisfy demands for \gls{carbon} and energy, thereby influencing the accumulation/loss dynamics of soil organic matter stocks and ecosystem carbon dioxide efflux \citep{xu2017}. The efficiency with which microorganisms convert available organic substrates into stable, biosynthesized products, is a critical step in ecosystem \gls{om} cycling. \gls{cue} is defined as the ratio between the amount of substrate-C incorporated into microbial biomass and the amount of substrate-C consumed. This measure is a common indicator for overall substrate use efficiency of the soil. The relevance of microbial efficiency spans scales ranging from a single microbial cell to entire ecosystems and variation in \gls{cue} has been linked to substrate biochemistry, thermodynamic and genetic capacity of the cell and microbial community structure and activity\citep{kallenbach2019, soares2019}. Other approaches have explored \gls{cue} from the ecosystem perspective, studying the effects of altered efficiency on the mediation of ecosystem services such as C sequestration. A considerable number of different experimental settings, methods and definitions have been used for the evaluation of \gls{cue} over the years, depending on the discipline and ecological framework in which \gls{cue} was examined. \citet{geyer2016} suggested a classification of \gls{cue} based on the ecological and temporal factors associated with the different definitions of \gls{cue}. The ecosystem perspective of carbon use efficiency (\gls{cue}$ _E $) as defined by \citet{geyer2016}, integrates drivers originating from population and community scales, as well as time-dependent factors such as biomass turnover and the recycling of necromass and exudates that occur external to the cell. Due to the relatively longer time periods associated with it (days to weeks), \gls{cue}$ _E $ reflects, on top of microbial community factors, environmental variables such as soil mineralogy, surface area and in fact any environmental factor that may exert influence on the turnover of microbial biomass and recycling of organic substrates. In this work, \gls{cue} will mostly be considered in the ecosystem context corresponding to \gls{cue}$ _E $.\\
	The drivers for efficient decomposition and cycling of organic carbon, in an ecosystem context, are evidently complex.
	Soil features such as mineralogy and aggregate structure may have a significant effect on \gls{cue}, particularly when microbial turnover and recycling of organic substrates are considered. For example, \citet{kravchenko2019} found the abundance of soil pores of size category 30 - 150 $ \mu m $ to be pivotal for the efficient stabilization of carbon through microbial activity. These pores act as sites for enhanced microbial activity and processing of new C input as they provide optimal water and oxygen supply. High abundance of such pores was found to ensure a large proportion of the soil matrix in close proximity to active microbial communities. In this way, microbial decomposition products and necromass have a greater chance of being stabilized in the soil through interaction with mineral surfaces and physical protection within soil aggregates, rather than being lost by mineralization.
	\citet{tian2015}, examined the effect of aggregate disruption on glucose decomposition and found significantly lower \gls{cue} for crushed, compared with intact macroaggregates. They concluded that aggregate structure must have an important role in shaping the dynamics of microbial decomposition of input C.

\section{Stabilization of \gls{som} closely connected with aggregate formation}

	The above mentioned works, along with a  number of other recent studies have made it evident that soil aggregation and the spatial organization resulting thereof, are central elements determining the fate of new input C. At the same time, the stabilization and incorporation of organic materials as part of \gls{som}, serve as a primary mechanism in the processes of aggregate formation. As noted earlier, newly recognized mechanisms of \gls{som} formation and preservation, include the physical and chemical protection of \gls{om} within and as a part of soil aggregates. Physical protection entails the inaccessibility of certain soil pores to micro-organisms and exoenzymes, making them optimal sites for the preservation of \gls{om}. This can result for example from actual size restrictions or the occurrence of conditions such as low oxygen availability which may preclude microbial utilization of \gls{om}. Chemical protection involves an array of organo-mineral interactions that can render \gls{om} inaccessible for microbial decomposition. By being chemically bonded to mineral surfaces, organic substances aid the formation of soil aggregates, acting as a gluing agent, rearranging soil particles into micro and macro aggregates \citep{six2002}. Thus, incoming \gls{om} (from plant origin or amendment source) acts as both the source of \gls{som} stocks and a major drive in the formation of soil structure, providing the platform for further stabilization and preservation of new incoming \gls{om} \citep{mccarthy2008}.\\
	While aggregate structure can enhance the stabilization of \gls{som}, the biological contributions to aggregate formation and stability are dependent on a supply and turnover of \gls{som} by micro-organisms. If the supply of \gls{om} is restricted, the aggregate stability declines and \gls{som} persistence is diminished. Therefore, the maintenance of stable aggregates  depends to a large extent on the turnover of \gls{som} in aggregates to sustain soil microorganisms \citep{dungait2012, golchin1994}.

%todo rephrase this?
\section{Labile organic matter}

	Labile, light weight organic substrates are increasingly recognized as having a major role in the formation of stable \gls{som}.
	Root derived inputs and specifically root exudates, constitute a significant source of such readily available organic carbon substrates.  It is assumed that about 20\% of all $ CO_2 $ assimilated in crop plants is deposited by roots into the soil as labile substances, generally referred to as rhizodeposits, and this percentage can be as much as 30-40\% in perennial grasses and other perennial plants \citep{kuzyakov2004, pausch2013, pausch2018}. Root derived carbon is much more likely to be incorporated into the more stable fractions of \gls{som} than above ground plant residues \citep{austin2017, kong2010, puget2001, rasse2005}.\\
	Rhizodeposition vary extensively in both composition and output rate according to plant species, physiological state and environmental conditions \citep{dennis2010, pausch2018, jones2004}. Furthermore, although it has a fundamental role for organic matter dynamics in the soil, it’s quantification under field conditions remains limited \citep{pausch2018}. Estimations of the rate of exudation for Maize plants for example, can vary extensively\citep{nguyen2003, pausch2018}.

\section{Dissolved Organic Matter}

	\citet{zsolnay2003} proposed the generic use of the term \gls{dom} when referring specifically to the organic matter truly dissolved in situ. Other soil solutions obtained through extraction with batch or percolation methods, such as the \gls{weom} or \gls{hweom} represent the \textit{potentially} soluble organic matter \citep{marschner2003}. \gls{dom} is a complex mixture of different components that can vary widely in both its composition and concentration depending on soil conditions  as well as on the method of extraction\citep{bolan2011}. \gls{dom} is often  correlated with \gls{som} stocks and indeed a number of relatively recent studies have shown \gls{som} to be the prime source of \gls{dom} as opposed to  recent plant derived \gls{om} \citep{malik2013, kaiser2012}. \citet{malik2013}showed that  even after a very high plant biomass input, plant-derived carbon in \gls{doc} was less than 40\%.\\
	The mechanisms and controls over \gls{som} decomposition processes, including solubilization  into the \gls{dom} pool are numerous, with both biotic and abiotic factors playing a significant role\citep{kalbitz2000, bolan2011}. Studies have shown that the microbial population is highly involved in the production and consumption of \gls{dom} \citep{marschnerp2002, malik2013, guggenberger1998}.  \gls{dom} has been characterized as being in a dynamic equilibrium with the solid phase \gls{som} \citep{roth2019, kaiser2012},  with the transfer of microbially processed organic molecules back and forth between solid and soluble phases controlled to varying degrees by  direct (exoenzymes) \citep{guggenberger1998},  and indirect(removal of solublized \gls{om} thus increasing diffusion gradient) solubilization of mineral associated \gls{om}. Additionally, solubilization can be affected by microbial processing of recent ( mostly plant source) \gls{om}, making it more available for interaction with mineral surfaces \citep{kalbitz2003, kalbitz2008}. These features of the \gls{dom} pool, position it as an intermediate \gls{om} pool, through which \gls{som} is made available to microbial consumption. Thus, \gls{dom} can provide valuable clues for microbialy mediated \gls{som} processes. \\

\section{Hot Water Extractable Carbon and Hot water extractable carbohydrates}

	\gls{hwec} represents a pool of potentially dissolved \gls{om}, less available than \gls{weoc} yet of a more biodegradable and labile nature \citep{chantigny2014, leinweber1995, gregorich2003}. \gls{hwec} is regarded as being weakly bonded to mineral surfaces or physically protected within soil aggregates and is a source for microbial consumption \citep{zakharova2015, leinweber1995}.
	\gls{hwec} is considered a  sensitive indicator of changes in soil management and \gls{som} dynamics, often responding to changes much faster than total organic carbon measures \citep{ghani2003}. \gls{hwec} is correlated with a number of important biochemical soil properties, among them, \gls{toc}, \gls{mbc} and aggregate stability \citep{hamkalo2014}.
	a considerable fraction of \gls{hwec} is composed of Carbohydrates \citep{leinweber1995, balaria2009}. \citet{ghani2003} found between 40-50\% of \gls{hwec} was in the form of carbohydrates in a series of grassland soils.
	\gls{hwes} were found to  closely reflect changes in aggregate stability \citep{haynes2005, yousefi2008, leguillou2012} and it is suggested that they have a major role as binding agents in the formation of soil aggregates.
	gls{hwes} are considered to be mostly of microbial origin \citep{haynes1993, debosz2002}, yet the proportion of unprocessed plant derived constituents can vary, depending, at least to a certain extent, on the level of microbial activity and the stability of the soil fraction from which carbohydrates are extracted. \citet{puget1998}, \citet{jolivet2006} and \citet{bock2007} all found microbially derived carbohydrates to be preferentially bonded to the most stable soil size fraction (caly + silt), indicating both the importance of certain  carbohydrates in stable aggregate formation and the fact these carbohydrates are mostly of microbial origin.
	Considering its close relation with aggregate formation and microbial growth and production, it is plausible that \gls{hwec} act as a substantial factor in  the dynamics of \gls{som} stabilization, at least for short term processes.

\section{Effect of management history on the long term dynamics of \gls{som} properties}

 	\subsection{Fertilizers and \gls{om} amendments}

 		The application of organic and inorganic amendments is a common practice in agricultural soil management. Long-term application of organic amendments, exclusively or in combination with mineral fertilizers, has been shown to increase \gls{toc} \citep{heinze2010, yang2012} as well as \gls{mbc} \citep{luo2015, marschner2003a}and aggregate stability\citep{ huang2010, yu2012} compared with a non-amended or mineral-only fertilized soil.
 		studies examining the effect of long term organic-only compared with mineral-only fertilization had  also shown clear evidence of a beneficial effect of organic fertilization on \gls{toc}\citep{santos2012, luo2015}, while the effects on more sensitive parameters such as \gls{mbc}, \gls{hwec} and \gls{ags} are sometimes less consistent \citep{albiach2001, mahmood1997}. It should be noted that results for this kind of comparison often depend on the choice of organic or mineral amendment as well as other management factors.    
 		For example, Liu et al. (2013) found  higher \gls{mbc} in long term manure compared with Urea-N fertilized soil, while no significant difference in \gls{mbc} was observed when a Phosphorus fertilizer was added to the Urea-N treatment. Additionally they estimated the annual total carbon input from crops as well as amendment carbon and these estimates had shown a {\raise.17ex\hbox{$\scriptstyle\mathtt{\sim}$}}35\% more input carbon in the manure treatment compared with the N-only treatment, not including the carbon input from manure amendment itself (i.e, in-field plant derived carbon). This hints at the importance  of plant productivity enhanced by organic or inorganic amendments, in determining the  effect of fertilization on soil biochemical parameters.\\
 		Besides management effects, \gls{som}  related parameters may also be subject to high seasonal variability in agricultural systems, depending on environmental changes as well as management interventions, further complicating the definite identification of long term fertilization effects \hiddenTxt{citation needed}.
 		Diallo-Diagne et al. (2016) found higher \gls{mbc} in manure fertilized soil compared with Urea-N fertilized soil in the beginning of the growing season, yet these differences faded at flowering stage and at harvest. Other works have also found significant temporal variations in \gls{mbc} between years \citep{kaiser1995} and plant physiological state \citep{jat2020}.\\
		It is evident therefore, that the effects of organic compared with mineral fertilizationin agricultural systems, with regard to the accumulation of \gls{soc} or the enhancement of other \gls{som} pools such as \gls{mbc}, \gls{weoc} and \gls{hwec}, are not entirely clear.

	\subsection{Effect of arable land management on \gls{som} properties}
%todo find more references for adverse effect of nitrogen fertilization, especialy with regard to reduced rehizodeposits or reduced carbon stabilization...
		Arable land management, and particularly intensive, modern day agricultural crop management practices, are well recognized as a serious environmental concern, along with its obvious contribution to human economy and well being.
		Modern agriculture is at the same time an invaluable resource for the earth’s population and a growing concern among researchers and environmentalists due to the numerous adverse effects it entails, such as, pollution and eutrophication of natural water bodies \citep{liu2012,rabalais2002}, greenhouse gas emissions\citep{vermeulen2012}, loss of biodiversity \citep{laurance2014} and soil degradation\citep{grunwald2011}.
		Because of these two contrasting effects of agricultural intensification, it is important to continuously evaluate the benefits of different agricultural management schemes with reference to more extensive agroecosystmes as well entirely non-cultivated land which are often not as susceptible to soil degradation/environmental damage.
		Natural, non-cultivated soils are usually characterized by minimal physical disturbance, little or no chemical inputs and an all year round soil cover (either live plants or dried stalks of annual plants), while arable soils are mostly subject to higher degrees of physical and chemical soil disturbance.
		these characteristics of cultivated and non-cultivated soils are general, and clearly large variability can be expected within these broadly defined features. For example, differences in vegetation type and plant coverage are evidently expected for non-cultivated soils, depending on the specific climatic and environmental conditions prevailing. For agriculturally managed soils, variability may be much higher considering the wide range of practices common in arable land management.
		Nonetheless, a large body of evidence suggests that land conversion of undisturbed, non-cultivated soils into arable soils can cause significant reductions in \gls{som} stocks \citep{ashagrie2007, spaccini2001, ogle2005}.
		native land conversion has also been shown to cause reductions in aggregate stability \citep{bongiovanni2006, cambardella1993,  elliott1986}, microbial biomass \citep{mganga2016, soleimani2019} and microbial diversity\citep{monkai2018}.
		the underlying drivers for these reductions in \gls{som} and related properties, through native land conversion, can be numerous and complex. Tillage, fertilization, reductions in plant cover, annuality vs. Perenniality and loss of biological diversity are some of the major factors that can adversely affect \gls{som} properties.
		Tillage is a common agricultural practice, often considered a major driver for soil degradation where native soils are shifted towards agricultural use (Murty 2002; Six 2002). The negative effect of tillage on \gls{soc} as well as aggregate structure is often observed in experiments comparing no-till or reduced tillage treatments with conventional tillage practices \citep{six1998, west2002}.
		Differences in tillage practice have  been shown to affect the \gls{mbc} of the soil \citep{jat2020, alvaro-fuentes2009, sun2011} as well as
		microbial community features such as composition, activity and functional diversity, with possible consequences for the dynamics of \gls{som} \citep{vangroenigen2010, govaerts2007}. These evidences prove the marked effect that physical disruption can have on soil biology and the processes of carbon retention and preservation. This is clearly demonstrated in the work of \citet{plaza2013}, in which significantly higher levels of mineral-associated carbon were observed in a long-term no-till treatment compared with conventional tillage treatment.\\
		Introduction of inorganic forms of Nitrogen as well as other inorganic elements, is another agricultural practice that can have adverse consequences for cultivated soils compared with non-cultivated soils.  \citet{khan2007}   presented extensive evidence demonstrating a net reduction in \gls{soc} as a result of long term nitrogen fertilization despite substantial increases in C return to the soil. \citet{egerton-warburton2000} found significant reductions in \gls{amf}  species richness, diversity, spore abundance and root colonization as a result of inorganic nitrogen pollution in a series of native scrubland soils in southern California. Numerous other studies have shown the adverse effect of mineral nitrogen fertilization on \gls{amf} diversity and abundance in agricultural settings \citep{bradley2006,treseder2004,zhang2019,corkidi2002}.  \gls{amf} are regarded as key players in the transfer and allocation of rhizodeposits, and are strongly associated with soil aggregation through fungal hyphe and the production of extracellular substances \citep{rillig2006,leifheit2014,rillig2002}. It is evident therefore that synthetic nitrogen fertilization can strongly affect soil carbon and aggregate dynamics through the restriction of \gls{amf} activity.\\
		Despite these well documented adverse effects of cultivation practices on \gls{som} properties, significant variation may exist in the response of soil to agricultural intensification, depending on climate factors such as precipitation and temperature. For example, a meta-analysis by Ogle et al (2005), found long-term cultivation had caused the greatest loss of \gls{soc} in tropical moist climates and lowest loss in temperate dry climate, demonstrating the variable nature of soil response to long-term cultivation in different climates.\\

\section{Effect of management history on short-term dynamics of \gls{som} properties}

		Apart from having a strong impact on \gls{soc} stocks as well on other \gls{som} related properties, management history can also have consequences for short-term dynamics of biochemical properties, when organic substrates are introduced into the soil, either as amendments or through plant deposition. Specifically, land use and fertilization management can affect the efficiency with which new organic inputs are decomposed and further stabilized as \gls{som} \citep{lee2014}. \gls{cue} has been shown to reflect long-term changes in soil management  while the underling mechanisms responsible for these variations in \gls{cue} are not entirely clear \citep{kallenbach2019}.
		\citet{kallenbach2015} applied the concept of \gls{cue} to try and explain higher long-term \gls{soc} accumulation in organically compared with conventionally managed soil, despite direct annual carbon inputs being slightly higher in the conventionally managed soil. Their findings showed significantly higher \gls{cue} in the organically managed soil and it was suggested that this higher \gls{cue} combined with higher microbial growth rates, resulting in higher microbial biomass and subsequently higher necromass, was a key factor in resolving the seemingly paradoxical phenomenon of increased \gls{soc} despite similar or even lower annual carbon inputs. Differences in \gls{cue} were mainly attributed to the inclusion of a leguminous cover crop in the organic management cropping cycle, which increased the chemical diversity of available organic matter and provided high quality (low C:N ratio), energy rich substrate, in this way favoring higher community-level \gls{cue}.
		\citet{roller2015} suggested that changing resource availability, caused by management, can cause a shift in microbial community composition, affecting \gls{cue}. Increased organic carbon availability, resulting in increased \gls{cue}, have been similarly suggested by other authors \citep{cotrufo2013}.
		Increased chemical as well as spatial heterogeneity of the soil environment may select for more efficient microbial communities \citep{pfeiffer2001, nunan2017}. Chemical heterogeneity could be achieved by diversification of organic inputs. Organic amendments, such as composted manure can provide a large number of different organic substances \citep{zbytniewski2005} and in this way greatly increase the  variety of substrates that the microbial population is exposed to. This could mean higher \gls{cue} for organically fertilized soils compared with mineral-only fertilized soil.
		As mentioned earlier, soil aggregation and the spatial arrangement of soil pores are major sources of heterogeneity and have been identified as important drivers of microbial efficiency\citep{kravchenko2019}.
		Tillage reduction or elimination can induce significant changes in aggregate stability and soil structural features \citep{alvaro-fuentes2009, barreto2009} and thus affect microbial efficiency.
		although much work has been done to expose the drivers and mechanisms that control microbial \gls{cue}, a lot of this work has concentrated on the short term effects of substrate quality and availability, on microbial species physiology and community dynamics. Less attention has been given to the effect of long term management decisions such as fertilization and the overall effect of land use on this essential component in \gls{som} cycling. Thus the nature and underlying drivers of these effects are still not clear.

\section{Short-to-medium term dynamics of organic substrate decomposition as essential components in understanding long term \gls{som} dynamics}

		Microbial efficiency is obviously affected by a large number of biological and environmental factors, and temporal considerations are important in defining the scope of factors expressed in microbial efficiency measures such as \gls{cue} \citep{kallenbach2019}. Incubation experiments, studying the decomposition of labile organic substrates, in the time range of days to months,  can demonstrate the effects on \gls{cue} of both physiological microbial population traits (usually associated with short-term periods of a few hours to a couple of days) as well as microbial community interactions with environmental factors such as soil physical and chemical features (usually more common in the days to months range) \citep{geyer2016, manzoni2018}. Moreover, medium term experiments are also important for the fact they can provide insight into the primary stages of labile substrate decomposition in the soil, as these processes are often mostly confined to these time ranges(days to weeks) \citep{blagodatskaya2011, schneckenberger2008, tian2015}. Although \gls{som} accumulation as well as other associated processes such as the development of complex aggregate formations, often require long time periods to materialize, especially in intensive agroecosystems \citep{grandy2007}, this gradual transformation is driven by short-term instances of organic matter decomposition and stabilization \citep{kuzyakov2015}. This is particularly valid for situations in which growing plants, through root deposition, are the major source of organic inputs. Root deposition and labile organic substances in general, are often significant sources of readily stabilized organic matter, and it is therefore important to consider their short to medium term decomposition dynamics with regard to long term trends in \gls{som} stocks.

\section{Research objectives}

	The main objective of this research was to examine the effects of long term, opposing management treatments, organic compared with mineral fertilization, on \gls{som} pools,  with reference to a non-cultivated soil. This is of particular importance in arid and semi arid climates, where the effect of agricultural intensification on \gls{som} properties may not be as straightforward as is often the case for soils in other climate zones.
	This main objective was split into two specific objectives, studying the effect of management history from two aspects:
	\begin{enumerate}
		\item \textbf{Effect on \gls{som} pools in non-amended samples.} for this purpose, the dynamics of non-amended samples was closely monitored in a set of two, short-term incubations, accounting for the normal fluctuations in \gls{som} properties and thus providing reliable indication of long term management effect.
		\item \textbf{Effect on the short-term dynamics of \gls{som} pools in response to labile organic input.} the short-term effect of labile organic amendment was examined in the different long-term treated soil samples, assessing any possible differences between these soil samples, upon specific sampling events as well as with regard to the overall trends.

	\numintertext{Additionally, a secondary objective was put forth:}

		\item \textbf{Examine the effect of successive labile organic substrate applications on the short dynamics of \gls{som} pools.} In accordance, three portions of labile substrate, applied within a two week period, were designed in  order to help identify possible mechanisms involved in short term carbon stabilization and carbon loss. This is in line with the notion that the interaction of substrate load with soil microbial and spatial features can have important implications for carbon dynamics.
	\end{enumerate}
	In exploring these objectives, particular consideration was given to microbial biomass size (gls{mbc})  and activity (CO2 respiration) as they were assumed to express the essential role of microbial populations in \gls{som} dynamics. Similarly, \gls{dom} dynamics, reflected in cold and hot water extractable \gls{om}, were given a significant focus as they are known to correlate with microbial features, especially those related to the active microbial biomass.  Considerable attention was also given to the aggregate stability parameter as an important indicator of soil structural features and to the \gls{hwes}, representing microbialy produced stable \gls{som} precursors, as well as for their important role as bonding elements in the formation of soil aggregates.
%todo add sentence about the use of Ergosterol analysis and its purpose.

\section{Research hypothesis}
%todo what is 'those described previously' in the end of the following paragraph?
	Considering the well recognized adverse effects of fertilization, tillage and other agricultural practices, I hypothesized that the lack of agricultural management in non-cultivated plots will have increased \gls{som} stocks and other related properties, compared with the cultivated soil plots. I assumed that the elimination of tillage and other physical and chemical disruptions associated with cropland management, would have warded off \gls{som} depletion, often observed for tilled soils, while inducing increased aggregate formation and biological activity (gls{mbc}, basal respiration)  compared with the cultivated soils through the extended presence of vegetative ground cover and increased above-ground diversity . These factors were also speculated to have a positive effect on \gls{cue} and the short term decomposition dynamics of recent organic input.\\
	Comparing the long term organically with the minerally fertilized soils, I expected to observe higher \gls{som} stocks as a result of long-term application of organic fertilizer, due to increased carbon input from fertilizer as well as increased aggregate formation that could have further enhanced the stabilization of input carbon.  I also expected increased microbial efficiency in processing recent organic substrate in the organically fertilized soil, mainly due to a more developed aggregate structure but also through increased chemical heterogeneity and possible long term effects of substrate availability causing shifts in microbial community composition.
	as for the effect of sequential labile carbon additions I expected consecutive applications to result in reduced microbial efficiency and would consequently lead to reduced levels of HWEsugars and \gls{ags}.  I hypothesized that the close proximity of three high rate labile organic input applications would produce conditions similar to those described previously, resulting in reduced microbial efficiency.
