	Organic matter is a key feature for soil health and fertility in both natural and managed ecosystems. In the last few decades there is an ongoing shift in paradigm regarding the nature of SOM and the conditions facilitating its formation and preservation. This emerging view stresses the dynamic nature of SOM, arguing that the potential preservation of SOM is essentially a soil ecosystem feature, rather than an inherent characteristic of the OM itself. It is now becoming more and more evident that optimal SOM management should also include the promotion of a viable soil biota, enabling the efficient processing of incoming OM fluxes. The current work employed soil samples of two contrasting long term organically (Org) vs. minerally (Min) fertilized plots from the Gilat Organic Plots, and another soil sample from an adjacent unmanaged soil (Unm). These soils were used to evaluate the effect of long-term fertilization and overall conventional field crop management, on both long-term changes in SOM pools as well as short-term dynamics of these SOM pools in response to substrate application. Additionally the effect of consecutive application of highly labile substrate on short-term carbon dynamics was examined. 
	A one week incubation and a 28 days incubation were set up, and various SOM pools as well as related soil features such as Aggregate Stability and Ergosterol concentration, were sampled for during the incubations. 
%	 Non-amended soils were analyzed to produce a baseline value of the different parameters measured during th incubation and this value was 
%%	Particularly, soil biochemichal featurs such as microbial carbon use efficiency and the dynamics of Hot and cold  Water Extractable carbon fractions were examined. This enabled the linking of long-term management practices with changes in the soil cappacity for effecient substrate decomposition.
	Total Organic Carbon was significantly higher in Org compared with Min and likewise, baseline values  for most other SOM pools were higher in Org over Min (though differences were mostly insignificant). Interestingly, Org - and Min to a certain extent - presented significantly higher baseline values than Unm, in TOC, basal respiration, MBC, WEOC, HWES and AS. 
	A consecutive application of a Model Root Exudate solution (MRE) during the 28 days incubation, resulted in a clear reduction in CUE for all three long term treatments. Likewise, the degree of labile soluble carbon (WEOC) removal from the soil system decreased substantially between consecutive MRE applications. The intensity of WEOC removal was deferentially expressed in the three LTTs with Org $  > $ Min $ > $ Unc, and these differences were very apparent. This demonstrated the effect       
	
	
%   continues and steady flux of labile carbonaceous substances (e.g plant root exudates)   
     