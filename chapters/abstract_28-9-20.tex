%!TEX root = ../main.tex	
	Soil Organic matter (SOM) is a key feature for soil health and fertility in both natural and managed ecosystems. In the last few decades there is an ongoing shift in paradigm regarding the nature of SOM and the conditions facilitating its formation and preservation. This emerging view stresses the dynamic nature of SOM, arguing that the potential preservation of SOM is essentially a soil ecosystem feature, rather than an inherent characteristic of the OM itself. It is now becoming more and more evident that optimal SOM management should also include the promotion of a viable soil biota, enabling the efficient processing of incoming OM fluxes.  \\
	The use of Organic amendments as part of agricultural land management can increase SOM and is often regarded as a sustainable alternative to mineral based fertilization. Nonetheless, organic fertilization particularly when combined with otherwise conventional field crop management should also be examined with reference to more natural, undisturbed soil ecosystems. This is especially important in arid and semi arid climates where soils are often with very low in organic matter and are prone to erosion and other soil degradation phenomenons.\\
	The current work compared soil samples of two contrasting, long term organically (Org) vs. Minerally (Min) fertilized plots, from the long term plots of the Gilat organic agriculture project (GOP) and another soil sample from an adjacent unmanaged soil (Unm). This long-term experiment included treatment with the relevant amendments for 5 years and then one year without fertilization and another year were the soil was laid bare ('no input + fallow' period).\\
	These soils were used to evaluate the effect of long-term fertilization and overall conventional field crop management, on both long-term changes in SOM pools as well as short-term dynamics of these SOM pools in response to substrate application. Additionally the effect of consecutive application of highly labile substrate on short-term carbon dynamics was examined. Particular attention was given to the effect of management history on the ecosystem efficiency of substrate-carbon use and potential stabilization in a short time frame.\\
	Previous studies in the GOP have seen significant increases in total organic carbon as well as other important SOM pools such as microbial biomass and microbial activity, in Org plots compared with Min plots. Similar works from different locations have found comparable results and the general benefits of organic amendments for enhancing SOM stocks as well as improving aggregate  structure, are well documented. Therefore I expected to observe significant increases in SOM pools and related features for Org compared with Min. In terms of the short-term dynamics, I expected the efficiency of substrate use to be higher for Org compared with Min, owing to the enhancement of different SOM pools and related soil features such as aggregate structure, which are hypothesized to have a significant effect on substrate use efficiency. 
	Conventional agricultural management is often reported to cause a reduction in various SOM pools and other related soil properties, when unmanaged soil is being utilized for crop production. Practices such as tillage, herbicide use and intensive chemical fertilization are well known as potential causes of soil degradation. Hence, I hypothesized that long-term conventional agricultural management would result in considerable reduction in SOM pools when compared with an unmanaged soil.\\
	Two incubation experiments were set up. The first, A 7 days Straw/Compost incubation which included soil samples from the two fertilized long term treatments and the addition of a single application of either straw or compost at the start of the incubation, as well as a control treatment without amendment. The second incubation, a 28 days Model Root Exudate(MRE) incubation, included all three soils (Org, Min and Unm). These were amended with a MRE solution at day 0, 7 and 14. In both incubations the following SOM pools throughout the incubations: Total organic carbon (TOC), Water Extractable Organic Carbon (WEOC), Hot Water Extractable Carbohydrates (HWES), microbial CO2 respiration and Microbial Biomass Carbon (MBC). The content of Ergosterol, a bioindicator for saprophytic fungi, was also measured. The Straw/Compost incubation included also the analysis of total organic carbon in a hot water extract and the MRE incubation included an Aggregate stability test besides the above mentioned tests.  The results from the control treatment in the MRE incubation were used to calculate baseline values for each of the measured parameters, allowing the comparison of the different long term treatments in terms of their effect on the normal status of these parameters. Microbial biomass and microbial respiration data were used to calculate a Carbon Use Efficiency measure.\\
	Total Organic Carbon (TOC) was significantly higher in Org compared with Min and likewise, baseline values  for most other SOM pools were higher in Org over Min (though differences were mostly insignificant). Interestingly, Org - and Min to a certain extent - presented significantly higher baseline values than Unm, in TOC, basal respiration, \gls{mbc}, \gls{weoc}, \gls{hwes} and \gls{ags}.\\ 
	MRE applications resulted in a clear reduction in CUE for all three long term treatments. Likewise, the degree of labile soluble carbon (WEOC) removal from the soil system decreased substantially between consecutive MRE applications. The intensity of WEOC removal was deferentially expressed in the three LTTs with Org $  > $ Min $ > $ \gls{unc}, and these differences were very apparent. No clear differences were observed between LTTs in terms of CUE.\\
	My results showed considerable enhancement of SOM pools after 5 years of compost application, when compared with mineral fertilization. Interestingly, this management history effect was noticeable despite the \textit{no input + fallow} period that preceded soil sampling for the current experiment, demonstrating the persistent effect of organic fertilization. The substantial increases in SOM pools and AS for Min and particularly Org, compared with \gls{unc}, were of special interest since this result is in disagreement with most \hiddenTxt{a bit to determind?} of the works on the subject. However, a number of studies have shown that in certain settings, for example such as an arid or semi-arid environment, cultivation may indeed enhance SOM pools when compared with a unmanaged soil. This was linked with the ecosystem's \gls{npp} and the fact that given the specific conditions in an arid or semi-arid climate, agriculturally managed soil can often sustain higher NPP compared with an unmanaged soil.\\
	The differences observed between LTTs with regard to WEOC dynamics, suggest that increases in total SOM stocks, MBC and AS, as a result of long-term organic fertilization, can have a substantial impact on the soil's capacity for removal of labile carbon from the dissolved fraction. Since no significant differences in CUE were noted between LTTs, it remains unknown whether this variability in WEOC removal was also related to differences in potential carbon stabilization. Nonetheless this outcome suggests increased MBC and microbial activity (as expressed in microbial respiration) and possibly an improved soil aggregate structure, can facilitate a higher decomposition capacity and increased potential for organic carbon stabilization as part of SOM.\\
	A conceptual model was proposed, describing the key factors involved in the potential short-term stabilization\textbackslash mineralization of incoming carbon. Based on this model, substantial decreases in CUE were related to a shift in the proportion between three key features of the soil system, i.e. size of the active microbial population, the quantity of labile substrate and the frequency of soil spatial sites of potential carbon stabilization. This analysis suggested that increased substrate load unaccompanied by similar increases in sites of potential carbon stabilization can result in substantial reductions in CUE. This reiterates the importance of a developed soil structure for the efficient use of incoming organic carbon.\\
	This study provided further support for the use of organic amendments to enhance SOM stocks and other important SOM pools. Setting the Compost treated samples against an Unmanaged soil, illustrated the variability of soil response to crop land management. This showed that cultivation, particularly when combined with organic amendments, such as composted animal manure, can improve soil fertility and health compared with an uncultivated soil.\\
	Furthermore, results from the MRE incubation indicated a substantial effect for management history in terms of the microbial capacity for utilization of labile carbon (WEOC) in the short-term of a few days. Although this result is confined to a short time range and does not explicitly include CUE or other carbon stabilization indicators, it is nonetheless a valuable clue to the effect of long-term fertilization and overall management on the soil's short-term reaction to incoming substrate. This can have important implications for long-term SOM dynamics under different land management schemes.  
	         
	
	
%   continues and steady flux of labile carbonaceous substances (e.g plant root exudates)   
     
