	Organic matter (OM) is a key feature for soil health and fertility in both natural and managed ecosystems. In the last few decades there is an ongoing shift in paradigm regarding the nature of Soil OM (SOM) and the conditions facilitating its formation and preservation. This emerging view stresses the dynamic nature of SOM, arguing that the potential preservation of SOM is essentially a soil ecosystem feature, rather than an inherent characteristic of the OM itself. It is now becoming more and more evident that optimal SOM management should also include the promotion of a viable soil biota, enabling the efficient processing of incoming OM fluxes. The current work employed soil samples of two contrasting long term organically (Org) vs. minerally (Min) fertilized plots from \myRed{an experimental site of a semi-desert climate} the Gilat Organic Plots, and another soil sample from an adjacent unmanaged soil (Unm). The long-term experiment included treatment with the relevant amendments for 5 years and then one year without fertilization and another year were the soil was laid bare ('no input + fallow' period). These soils were used to evaluate the effect of long-term fertilization and overall conventional field crop management, on both long-term changes in SOM pools as well as short-term dynamics of these SOM pools in response to substrate application. Additionally the effect of consecutive application of highly labile substrate on short-term carbon dynamics was examined. \myRed{Particular attention was given to the effect of management history on the ecosystem effeciency of substrate-carbon use and potential stabilization.}
	A one week incubation with either Wheat Straw or Kitchen Waste Compost, and another 28 days incubation with a \gls{mre} solution were set up. \gls{mre} was applied consecutively at $ T_0 $ and at day 7 and 14. Various SOM pools and related soil featurs (Aggregate Stability and Ergosterol concentration), were sampled for during the incubations.  
%	 Non-amended soils were analyzed to produce a baseline value of the different parameters measured during th incubation and this value was 
%%	Particularly, soil biochemichal featurs such as microbial carbon use efficiency and the dynamics of Hot and cold  Water Extractable carbon fractions were examined. This enabled the linking of long-term management practices with changes in the soil cappacity for effecient substrate decomposition.
	Total Organic Carbon (TOC) was significantly higher in Org compared with Min and likewise, baseline values  for most other SOM pools were higher in Org over Min (though differences were mostly insignificant). Interestingly, Org - and Min to a certain extent - presented significantly higher baseline values than Unm, in TOC, basal respiration, \gls{mbc}, \gls{weoc}, \gls{hwes} and \gls{ags}. 
	MRE applications resulted in a clear reduction in CUE for all three long term treatments. Likewise, the degree of labile soluble carbon (WEOC) removal from the soil system decreased substantially between consecutive MRE applications. The intensity of WEOC removal was deferentially expressed in the three LTTs with Org $  > $ Min $ > $ Unc, and these differences were very apparent. No clear differences were observed between LTTs in terms of CUE.
	My results showed considerable enhancement of SOM pools after 5 years of compost application, when compared with mineral fertilization. Interestingly, this management history effect was noticeable despite the \textit{no input + fallow} period that preceded soil sampling for current experiment, demonstrating the persistent effect of organic fertilization. The substantial increases in SOM pools and AS for Min and particularly Org compared with an unmanaged soil (Unm), were of special interest since this result is in disagreement with most of the works on the subject. However, a number of studies have shown that in certain settings, for example such as an arid or semi-arid environment, cultivation may indeed enhance SOM pools when compared with non-cultivated soil. This has been linked with the ecosystem's NPP and the fact that given the specific conditions in an arid or semi-arid climate, agriculturally managed soil can often sustain higher NPP compared with an unmanaged soil.
	The differences observed in WEOC dynamics between LTTs, suggest that increases in total SOM stocks, MBC and AS, as a result of long-term organic fertilization, had a substantial impact on the soil's capacity for labile soluble carbon removal from the dissolved fraction. Since no significant differences in CUE were noted between LTTs, it remains unknown whether this variability in WEOC removal was also related to differences in potential carbon stabilization. Nonetheless this outcome suggests increased MBC and microbial activity (as expressed in microbial respiration) and possibly an improved soil aggregate structure, can facilitate a higher decomposition capacity and increased potential for organic carbon stabilization as part of SOM.
	A conceptual model was proposed, describing the key factors involved in the potential short-term stabilization\textbackslash mineralization of incoming carbon. Based on this model, substantial decreases in CUE were related to a shift in the proportion between three key features of the soil system, i.e. size of the active microbial population, the quantity of labile substrate and the frequency of soil spatial sites of potential carbon stabilization. This analysis suggested that increased substrate load unacompanied by similar increases in sites of potential carbon stabilization can result in substantial reductions in CUE. This reiterates the importance of a developed soil structure for the efficient use of incoming organic carbon. This study provided further support for the use of organic amendments to enhance SOM stocks and other important SOM pools. Setting the Compost treated samples against an Unmanaged soil, illustrated the variability of soil response to crop land management, and showed that cultivation, particularly when combined with organic amendments, such as composted animal manure, can improve soil fertility and health. Furthermore, results from the MRE amendments indicated a substantial effect for management history in terms of the microbial capacity for utilization of labile carbon (WEOC) in the short-term of a few days. Although this result is confined to a very short-term and does not explicitly include CUE or other carbon stabilization indicators, it is nonetheless a valuable clue to the effect of long-term fertilization and overall management on the soil's short-term reaction to incoming substrate. This can have important implications for long-term SOM dynamics under different land management schemes.  
	         
	
	
%   continues and steady flux of labile carbonaceous substances (e.g plant root exudates)   
     