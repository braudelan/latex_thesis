% !TeX root = ../main.tex

% questions for asher
% 1. KWC, what is a better term for that?
% 2.

%todo include a section about modeling of respiration dynamics

\section{Soil sampling}
The soils for this experiment were sampled from the long term plots installed at Gilat in 2009 as part of the organic agriculture project (\gls{gop}), Bar-Tal et al (2017). The Gilat Research center is situated in the northern Negev(31$^\circ$20' N, 34$^\circ$41' E), the climate is a semi-desert (average precipitation $ 237_{mm} $) and the soil is a sandy loam loess, defined as a calcic Xerosol (Soil Survey Staff, 2010). Soils' main properties determined for replicated samples obtained from five different plots of the experimental field platform in June, 2010 are: Texture - Sandy loam (Clay - 24.7 $\pm$ 1.0\%, Silt - 35.2 $ \pm $ 1.6\%, Sand - 40.1 $\pm$ 1.7), TOC - 0.51 $\pm$ 0.16\%, TN - 0.08 $\pm$ 0.005\%, CEC - 16.7 $\pm$ 0.3 $ meq *
100g^{-1} $, CaCO3 - 18.3 $\pm$ 0.9\%, pH - 8.51 $\pm$ 0.06. The pH was determined in soil: distilled water mixture (1:4 w/w). Calcium carbonate content was determined using a calcimeter. The clay, silt and sand contents were determined using the hydrometer method based on Stokes’ law. The cation exchange capacity (CEC) was determined following Rhoades (1982).\\
The experimental platform included 4 treatments in 4 randomized blocks with five replicates (20 sub-plots on an area of ca.1 hectar). These four treatments were as follows: three levels of cow manure compost application at the rates of 20, 40 and 60 $m^3\cdot hectar^{-1}\cdot year^{-1}$ (equivalent to 1, 1.9 and 3.8 $kg^3\cdot dunam^{-1}\cdot year^{-1}$ respectively) and a control treatment (urea fertilizer). For the current experiment soil samples were taken from the 6$m^3$ compost treated plots and the urea treated plots. Additionally soil samples were taken from a long-term non-cultivated area outside the experimental grounds (unmanaged for at least 6 years, since the Gilat Organic Plots were set up).  Five soil samples (one for each block) were sampled from each treatment, sieved (\textless 2mm) and combined together to yield three compound samples. Thus, three \glspl{ltt} were represented by these compound samples: \gls{org}, \gls{min} and a \gls{unc}.
The annual crop rotation program, compost application schedule and all other elements were according to typical organic agriculture management standards in the area. The last fertilizer application was preformed on June 2015 and January 2016 for the \gls{org} and \gls{min} treatments respectively. The last harvest took place on 14/11/16 and from that time on the soil was laid bare. Soil sampling was done in October 2017 before the first rain.





\section{Preliminary experiment}

	A short term preliminary incubation was established in order to evaluate the behavior of the different LTTs under organic substrates application of differing lability.  Only ORG and MIN were used for this experiment.  The preliminary incubation included 2 treatments (referred to as \gls{stt}), application of \gls{str} (dried at 60 deg C overnight and ground fine) and the application of \gls{dwc} \hiddenTxt{This compost is from the 'Chava VeAdam' ecological farm in Modi'in, a small scale compost facility using waste products from the local kitchen}, both at a rate of 0.2\% Carbon ()weight based). Table \ref{amendments_properties_preliminary} show the basic properties of these substrates. A control treatment was also included with no substrate applied.

		\begin{table}[H]
\centering
\caption{basic properties of short term amendments in preliminary incubation}
\label{amendments_properties_preliminary}
\begin{threeparttable}

\begin{tabular}{l    cc}
\toprule
{Substrste} &    \gls{kwc} &   \gls{str} \\
\midrule
{C/N              } &   12.2 &  45.0 \\
{OM (1.72*TOC) (\%)} &   37.4 &     - \\
{DOC ($ g*kg^{-1} $)    } &  3.45 &  15.0 \\
{Dissol. C/N      } &    2.3 &     - \\
{pH               } &    7.2 &   6.8 \\
\bottomrule
\end{tabular}

\begin{tablenotes}
	\item[*] \scriptsize \gls{kwc}, kitchen waste compost; \gls{str}, wheat straw;  OM, organic matter; DOC, dissolved organic carbon;
\end{tablenotes}

\end{threeparttable}
\end{table}


	\noindent The experiment was done in 4 replicates with a total of 24 experimental units. This basic arrangement was duplicated five times to account for destructive samplings during the incubation period. 20 grams of soil were packed into small plastic containers, 120 ml in volume, {\raise.17ex\hbox{$\scriptstyle\mathtt{\sim}$}}6 cm in diameter and 7 cm high. The soil was wetted with distilled water up to 45\% of \gls{whc} and incubated for 48 hours before the start of the experiment in order to ensure uniform conditions. At the start of the experiment straw/compost was applied to relevant cups and thoroughly mixed with the soil after which all experimental units were brought to 60\% WHC using distilled water. The soils were incubated for one week in a set temp of 20 deg C in dark conditions. Plastic cups were caped with perforated seals allowing ample gas exchange. The following analyses were sampled for 24, 48, 96 and 168 hours after the start of the incubation: \gls{mbc} and \gls{mbn}, \gls{weoc}, \gls{hwec} and \gls{hwes},  and soil ergosterol content (ERG). Soil \gls{resp} was measured 2, 6, 12, 24, 48, 96 and 168 hours into the incubation period.


\section{Incubation with labile organic substrate}

	Based on results from the preliminary experiment, a similar experiment was planned with some adjustments. \gls{unc} was also included in this experiment, along with the two other LTTs.
	These three LTTs were used in a short term experiment in combination with a \gls{mre}, representing a common mixture of organic substrates available in the soil rhizosphere. The composition of MRE (table \ref{MRE_composition}) is derived from Traoré et al. (2000).
	20 grams of each LTT was lightly packed into plastic cups (described above), brought up to 50\% WHC with distilled water and transferred to an incubator for 1 week, at 25 deg C in the dark, to achieve uniform conditions. The soils were then incubated for another 28 days at 25 deg C in dark conditions with a weekly portions of MRE at a rate of 2200 mg C * kg soil-1  applied at the beginning of each week during the first three weeks ,amounting to a total of 6600 mg C * kg soil-1. Nitrogen applied as part of MRE solution amounted to 1.4\% of the carbon weight,  totaling 92.4 mg by the end of the incubation. Soil respiration was measured 2, 4, 10, 24, 72 and 168 hours after MRE solution was applied for every week of the first 3 weeks of incubation and also at incubation end. Samples for destructive analysis were collected 2 hours after the first MRE application and 24, 72, 168 hours after  each MRE application for every week of the first 3 weeks of incubation and at the end of the incubation. The following tests were applied for each of the destructive samplings: MBC/N, HWES, ERG and Water Extractable Organic Carbon (WEOC). \gls{toc}, \gls{ags} and \gls{tin} were measured at the beginning of the incubation (at the end of the pre-incubation), after 14 days and at the end of the incubation. \gls{ec} and pH were measured at the beginning and end of the incubation.


    
 \begin{table}[H]
    \centering
    \caption{MRE composition}
    \label{MRE_composition}
%the \hskip command inserts a blank space
    \begin{tabular}{l@{\hskip 1in}c} 
    	
        \toprule
        components of MRE & \% of total carbon\\
        \midrule
        Sugars\\
        Glucose           &  74.6\\
        \textit{Total}    &  74.6\\\\
        Amino acids
        Alanine           &  0.17\\
        Glutamic acid     &  0.34\\
        Glycine           &  0.19\\
        Lysine            &  0.21\\
        Methionine        &  0.21\\
        Proline           &  0.15\\
        Valine            &  0.15\\
        \textit{Total}    &  1.4\\\\
        Organic acids\\
        Citric acid       &  7.03\\          
        Lactic acid       &  3.64\\
        Malic acid        &  3.39\\
        Succinic acid     &  9.94\\
        \textit{Total}    &  24.0\\
        \bottomrule
                  
    \end{tabular}

\end{table}


\section{Analytical methods}

    \subsection{CO$_2$ respiration}

    	Prior to sampling, experimental units (plastic cups) were sealed using a lid equipped with a Teflon septa for several hours depending on the expected rate of respiration. 2 ml of gas samples were pumped out of sealed incubation vessels using sterile syringe and needle. Gas samples were injected into empty sealed glass vials and transferred for analysis of CO2 using a \gls{gc} ( varian 450-GC, Bruker Daltonics Inc) connected to an auto sampler(headspace autosampler - Teledyne, Tekmar HT3 Mason, Ohio (OH), USA). Analysis was carried out using a Thermal conductivity detector  with detector temperature at 200 C and filament temperature at 290 C. Gas components were separated using a Mol Sieve Capillary \gls{gc} Column (0.5m X 3.17mm), with helium as a carrier and supplemental gas. Column temperature was set to 50$ ^\circ $ C. $ CO_2 $  standards were used to construct a calibration curve.
    	Soil CO2 respiration rate was calculated in the following way:\\

   		\begin{align} % adding astrix(*) to the align argument will remove equation numbering
   		\triangle Area &= Area_{sample} - Area_{ambient}\\
   		\triangle CO_2 &= \frac{\triangle Area}{8.568} \label{calibrationcurve}
   		\end{align}

    	\begin{adjustwidth}{50pt}{50pt}
    		\begin{footnotesize}
    			\textit{Where $Area$ is the integrated area of the chromatogram peak for CO$_2$, generated by the program provided with the GC instrument. $ Area_{sample} $ and $  Area_{ambient} $ are therefore the area of peak for a specific sample and for the ambient  $ CO_2 $ concentration respectively.  $\triangle CO_2$  is the increase in $ CO_2 $ concentration in the experimental cups above the ambient concentration. Equation \eqref{calibrationcurve}, used for calculating  $\triangle CO_2$ is derived from preliminary calibration experiments.}
    		\end{footnotesize}
		\end{adjustwidth}

    \subsection{MBC\textbackslash{}N and WEOC}

    	Microbial biomass was estimated using the fumigation-extraction method (Vance et al., 1987), with minor modifications. 20 g of moist soil ({\raise.17ex\hbox{$\scriptstyle\mathtt{\sim}$}}50\% WHC) were split into two portions of 10 g (fumigated and non-fumigated). 10 g portions dedicated for fumigation were placed in 25 ml beakers and fumigated with chloroform (CHCl3; HPLC grade $ > $ 99.9\% purity, Sigma-Aldrich). Fumigation was carried out within a glass desiccator with internal diameter of 300 mm. A 50-ml glass beaker filled with {\raise.17ex\hbox{$\scriptstyle\mathtt{\sim}$}}40 ml chloroform (CHCl3; HPLC grade $ > $ 99.9\% purity, Sigma-Aldrich) and some glass boiling beads (diameter 3 mm; Sigma-Aldrich) were placed in the midst of the soil-containing beakers.  Both fumigated and non-fumigated soil samples were then transferred to 50-ml polypropylene centrifuge tubes (Greiner) and extracted for 30 min with 40 ml 0.5M K2SO4 solution on an orbital shaker at 100 rev min 1. Centrifugation (4500 rpm for 20 min) and subsequent filtration of the supernatant using 0.45 mm pore size carbon membrane filters (Durapore PVDF, Millex-HV by Merck Millipore Ltd.) followed. Soil extracts were then analyzed for total organic carbon and total nitrogen (TN) using a total organic carbon analyzer (Shimadzu, TOC-V CPN). Total organic carbon measured in non-fumigated samples was regarded as total WEOC. Flush values (delta of carbon available after fumigation) were calculated as the difference between TOC/TN values of fumigated samples minus non-fumigated samples. These values were then used to calculate MBC/N as follows:\\


    	\begin{align}
    	Flush &= TOC_{fum} - TOC_{non\_fum}\\
    	MBC   &= Flush * K_{EC}
    	\end{align}

    	\begin{adjustwidth}{50pt}{50pt}
    		\begin{footnotesize}
    			\textit{Were $ TOC_{fum} $ and $ TOC_{non\_fum} $ are the TOC values of soil extracts of fumigated and non-fumigated soil respectively and $ K_{EC} $  is the correction factor compensating for the incomplete extractability of fumigated biological material as suggested by Joergensen et al. (1996).}\\
    		\end{footnotesize}
    	\end{adjustwidth}


    \subsection{Hot water extractable sugars}

    	Soil samples that were used as the non-fumigated portion in fumigation extraction procedure were sequentially used for hot water extraction in the following manner: the supernatant left in the centrifuge tube from the previous extraction was discarded and 40 ml of distilled water were added to the sediment. The tubes were then placed in a water bath heated to 80$\circ$ C and the temperature was kept constant using an electric heater. Tubes were left in the bath for 16 hours, centrifuged (4500 rpm for 20 min) and the supernatant filtrated using 0.45 mm pore size membrane filters (Durapore PVDF, Millex-HV by Merck Millipore Ltd.). The extracts were kept in cold storage for further analysis.
    	Total Hot Water Extractable organic Carbon (HWEC) was measured with a  total organic carbon analyzer (Shimadzu, TOC-V CPN) as described above for MBC and WEOC.
    	Carbohydrates in hot water extract were measured using the phenol-sulfuric acid method(Dubois et al., 1956). The basic principle of this method is that carbohydrates, when dehydrated by reaction with concentrated sulfuric acid, produce furfural derivatives. Further reaction between furfural derivatives and phenol, develops detectible color. The procedure was conducted as follows, 2 ml of the extract were placed in a glass tube and 50 $ \mu L $ of a liquefied phenol (C6H6O; $ \geq $89.0\% purity, Sigma-Aldrich) were added to the extract. Immediately after that 5 ml of concentrated sulfuric acid (H2SO4, 95.0-98.0\% purity, ACS reagent grade, Sigma-Aldrich) were rapidly added. The mixture was then left to rest for 10 min after which it was vortexed for 30 seconds and then cooled to room temperature. One replicate from every set of four soil-treatment replicates was prepared as a soil blank in the same manner described above, omitting the addition of phenol. The absorbances of the samples were measured at 485 nm against soil blanks as described. Glucose (C6H12O6; $ \geq $99.5\% purity, Sigma-Aldrich) standards were prepared in the same way, measured against distilled water and the results were used to construct a calibration curve.


    \subsection{Ergosterol}
%todo include Nadia Buchnovsky in the acknowledgement
%todo ask Nadia about the system name of HLPC for Ergosterol detection
    	The specific protocol used in this work to prepare and analyze ergosterol concentration was determined by a modified method based on a number of relevant works (Ruzicka et al., 2000; Gong et al., 2001; Wallander et al., 2013; Beni et al., 2014). Preliminary tests were conducted in order to establish the optimum parameters for extraction and measurement of ergosterol in the context of this work. Among the parameters that were optimized are: extraction solvent (water/methanol/methanol+KOH), shaking time for extraction ( 2, 6, 10 and 14 hours) and the wavelength of excitation for detection of ergosterol in the extract. The formulated protocol is as follows: Ca. 0.4 g of soil were shaken (120 rpm) in glass vials with 3 ml of pure methanol (CH3OH; HPLC grade $ > $ 99.9\% purity, Sigma-Aldrich ) for 14 h at 28$ \circ $ C. The vials were then centrifuged ( 2000 rpm, 30 min) and filtered (0.45 $ \mu m $ pore size, hydrophobic PTFE, Merck Millipore Ltd.). Ergosterol concentration in the extract was analyzed using HPLC, with Kinetex 5u EVO C18 column (phenomenex), methanol as the mobile phase, the flow rate 1 ml * min-1 and the temperature set to 30$ ^\circ $ C. Ergosterol was identified by UV detection at 282 nm. An Ergosterol standard was used for calibration.


    \subsection{Total organic carbon}

    	Measurement of TOC was performed with a C/N elemental analyzer (Flash EA 1112, Thermo Fisher Scientific). This analysis involves heating the sample to very high temperature (up to 1800$\circ$ C) for a few seconds causing the components of the sample to transfer into an elemental gaseous state. The different gases are then separated in a column and measured with thermal conductivity detector. Prior to analysis, inorganic carbon was removed from samples by placing them in an acidic atmosphere overnight, subsequently evaporating any leftover acid in a ventilated hood followed by drying in an oven at 80$\circ$ C overnight.


    \subsection{Aggregate stability}

    	An automatic wet sieving apparatus (Eijkelkamp Agrisearch Equipment, Giesbeek, The Netherlands) was used for measuring aggregate stability. Roughly 2.5 g of soil aggregates (1-2 mm in size) were placed in plastic baskets equipped with a 250 $ \mu m $ sieve. Aggregates were then gently wetted with 1 ml of distilled water so as to prevent slaking, before commencing with the test itself. The sieves are  placed into the apparatus and lowered into metal cans filled with \gls{dw}. The machine is then turned on so that the sieves are gently moved up and down in the water for a fixed time (3 min). Unstable aggregates break down and pass through the sieves after which the apparatus is paused and the baskets are raised out of the water. Any remaining material in the baskets (I.e stable aggregate larger than 250 $ \mu m $) was handled in a similar way as before, except a new set of cans was now filled with a 2 g/L calgon solution (sodium hexametaphosphate, 96\% purity, Sigma-Aldrich) in order to break down all remaining aggregates. The calgon/DW filled cans were then oven dried overnight @ 105 C and  then let cool and weighed. The weight of material collected in the calgon/DW filled cans was used to calculate the percent of stable aggregates $ > $250 $ \mu m $ in the following way:\\

    	\begin{align}
    	Total_{dis} &= Total_{initial} - residue\\
    	\%_{stable} &= \frac{Calgon_{dis}}{Total_{dis}}
    	\end{align}

    	\begin{adjustwidth}{50pt}{50pt}
    		\begin{footnotesize}
    			\textit{where $ Total_{initial} $ is the initial weight of aggregates placed in the collecting can; $ residue $ is any resistant material left in the sieves after the Calgon wash (mainly fragments of organic matter  or rock material) and $ Calgon_{dis} $ is the weight of material collected in the Calgon filled cans.
    			}\\
    		\end{footnotesize}
    	\end{adjustwidth}

    \subsection{Mineral nitrogen}

    	soil samples were extracted by shaking 3 g of oven dry and ground up soil with 15 ml of a 74.55 g/L KCl solution for 40 min at 100 rev * min-1. The samples were then centrifuged and filtered. Total mineral Nitrogen ( excluding NO$_2$-N ) was analyzed with a photometric analyzer ( Thermo Scientific Gallery , Thermo Fisher Scientific).

    \subsection{pH and EC}

    	A 1:5 soil extraction was used for pH ( PHM92 RadioMeter Copenhagen ) and EC ( CON510 Eutech Instruments ).


\section{Calculations and statistics}

    \subsection{Statistical analysis}

  	 Statistical analyses and data processing was performed using Python’s SciPy library (SciPy 1.0: Fundamental Algorithms for Scientific Computing in Python). Significance was computed with Python’s statsmodels package (Seabold 2010) , using independent t-test in a multiple pairwise comparisons test with the Holm-Bonferroni method for p-value corrections (Stephanie, 2016). a significance level of $\sigma$  = 0.05, was used for all tests. \\
%    	\myGreen{Correlations between p	airs of \gls{som} properties were examined through a linear model regression analysis using the statsmodels package. The  $ r^2 $ coefficient was used to asses...} \\

    \subsection{Baseline}

    	In order to evaluate the effect of long term management treatments (LTTs) on different soil features, a baseline value was calculated for every soil feature in each LTT. The entire set of replicates from control samples across the incubation period was pooled together and the mean value was calculated and taken as the baseline value for the corresponding parameter-LTT pair. In-week sampling events between week ends were omitted to eliminate the effect of water additions at the beginning of each week. This yielded a total of N=20 replicates for each parameter-LTT pair ( except for AS where 12 replicates were available).

    \subsection{Carbon use efficiency}

		When applied in the context of short-term incubations with organic carbon sources \gls{cue} is commonly defined as follows:\\

		\begin{align}
		\gls{cue} = \frac{MBC_{substrate}}{(MBC_{substrate} + CO_2-C_{substrate})}\label{eq:cue}
		\end{align}

  	 	\begin{adjustwidth}{50pt}{50pt}
  	 		\begin{footnotesize}
  	 			\textit{where the subscript $ substrate $ signifies the Carbon fraction of the respective component (I.e either MBC or CO$ _2 $) that is of substrate origin.
  	 			}\\
  	 		\end{footnotesize}
  	 	\end{adjustwidth}
    	This definition of \gls{cue} evidently requires the use of carbon labeling or other techniques that would enable the definite discrimination of substrate-C from native \gls{som} derived C so that strictly substrate-originating C is reflected in $ MBC_{substrate} $ and $ CO_2-C_{substrate} $.
    	My work did not include any carbon labeling, Nonetheless I decided to include calculations of weekly \gls{cue} based on cumulative CO2 and the change in MBC. The validity of using \gls{cue} in this manner was based on two major assumptions:\\
    	\begin{enumerate}
    		\item By the end of every period in which \gls{cue} is calculated for, the large majority of substrate-C had undergone microbial uptake, that is to say, all substrate-C is either potentially recoverable as MBC or otherwise as microbial products or necromass.
    		\item The proportion of primed carbon from newly formed MBC (MBC$ _{new} $) or from the total $ CO_2 $ pool for the respective period is negligible.
    	\end{enumerate}
    	Satisfying these basic assumptions ensures that the measured total MBC and CO$ _2 $  for a given time period approximately represent the entire substrate-C fraction allocated into these respective C pools and that the calculation of \gls{cue} based on this data is appropriate. Thus, \gls{cue} was calculated according to Eq. \ref{eq:cue}, using the total weekly change in MBC and weekly cumulative CO$ _2 $ as  $MBC_{substrate} $ and $ CO_2-C_{substrate} $ respectively.




    \subsection{treatment effect}
        the effect of short term amendments, both in the preliminary and the main incubation, was calculated for each sampling event, by subtracting the corresponding control value for the same sampling event. Values normalized in this way are designated by the subscripted suffix \texttt{Net}.


%         \begin{align}
%         Treatment\ Effect = Amended_{T} - Control_T   \frac{MBC_{substrate}}{(MBC_{substrate} + CO_2-C_{substrate})}\label{eq:cue}
%         \end{align}

%          using two different approaches, according to the soil parameter and the specific relevance of each approach.
%     	These two methods are as follows, where T$_i$ is a given sampling event: \\


%    		\begin{enumerate}
%    			\item \textbf{Control\_normalized (Cn)}, treatment effect is evaluated as the difference between values of treated samples at T$_i$ and the values of control samples on the same T$_i$. The mean value of all replicates of control samples at T$_i$ (of a given LTT) was subtracted from the value of each treated replicate. The mean of these differences was regarded as treatment effect.\\
% %   			\myGreen{maybe add a math representation?}
%    			\item  \textbf{Initial\_normalized(In)},  treatment effect is evaluated as the difference between treated samples at T$_i$ and the initial value (T$ _0 $) of control samples. Here again, the mean value of control replicates was subtracted from each treated replicate and the mean of these differences was regarded as treatment effect.
%    		\end{enumerate}

    \vspace{2cm}


    \subsection{Error propagation}

    	Whenever arithmetic operations were performed over mean values, their respective standard errors were propagated in the following manner::\\
    	\begin{itemize}
    			\item for additive operations -\\

    			\begin{align}
					Result\ s_{\bar{x}}    &=    \sqrt{s_{\bar{x}\ A}^2\ +\ s_{\bar{x}\ B}^2 }
					\intertext{\item for multiplicative operations}
    				r\_s_{\bar{x}} 	       &=	 \frac{s_{\bar{x}}}{\mu}\\
    				Result\ r\_s_{\bar{x}} &=  \sqrt{r\_s_{\bar{x} A}^2 + r\_s_{\bar{x} B}^2}\\
    				Result\ s_{\bar{x}}     &= Result\ r\_s_{\bar{x}} * Result
    			\end{align}

    	\end{itemize}

   		\begin{adjustwidth}{50pt}{50pt}
    		\begin{footnotesize}
    			\textit{where $ A $ and $ B $ are mean values computed with each other either additively or multipicatively and $ Result $ is the result  of the operation. $ s_{\bar{x}} $ is the Standard Error of the mean ($ \mu $) and the prefix $ r $ stands for relative.
    		}
    		\end{footnotesize}
    	\end{adjustwidth}
